
\documentclass[twoside]{article}
\usepackage{fullpage}
\usepackage[pdftex]{graphicx}
\usepackage{wrapfig}
\usepackage{amsmath}
\usepackage{hyperref}
\usepackage{sectsty}
%\sectionfont{\fontsize{13}{15}\selectfont}
\usepackage{fixltx2e}
\usepackage{fancyhdr}
\pagestyle{fancy}
\fancyhead{}
\fancyfoot{}
\renewcommand{\headrulewidth}{0pt}
\fancyfoot[LO]{\emph{Hall - CSI 370}}
\fancyfoot[LE]{\emph{Research 1 - Project}}
\fancyfoot[R] {\thepage}
\newenvironment{code}{\fontfamily{lmtt}\selectfont}{}
\date{}
\begin{document}
\title{CSI 370 Computer Architecture \\ Research 1 - Project}
\author{Brian R. Hall \\
Champlain College\\
hall@champlain.edu}
\maketitle
\renewcommand{\labelitemi}{$\diamond$}
\noindent It is time to start thinking about projects. This activity/assignment
requires you to write a one-page overview of your intended project. This will also
be the beginning of your technical report that you submit alongside your
implementation. All documentation should be written in {\LaTeX}. \textbf{The first
submission, ``And i change this specific secition of the file to say something new, truly whatever i want it to, no one can stop me i have all the power here,'' should begin explaining the what, why, how, and
potential challenges of your project on a macro scale. The one-page submission is
due one week after assigned.}
Students may work independently or in pairs. The project is flexible and has two
options: (1) use a combination of provided hardware kits and microcontroller
programming to build a device, or (2) implement an appropriate application or
portion of an application using Assembly. \textbf{The project must be unique to
this class.}
As you consider project ideas, keep scope in mind. Choose something that is
challenging, but doable given the timeframe. Projects will be due the last day of
class and may be presented to the class. \\
\textbf{Option 1 - Arduino/Hardware Project:}
\begin{itemize}
\item Must be unique in the sense that it cannot be an exact implementation
of a tutorial found online from the Arduino site or elsewhere. You can use
tutorials as a starting point if needed, but the final product must go beyond
following a guide.
\item You are allowed to use the Ardunio C-language variant for most of the
software implementation for the project, but at least one code section or function
must be written using the microcontroller's instruction set. As part of your
technical report, you must elaborate on the ``Assembly'' section of your code and explain the instructions and their function.
\item Arduino Uno is based on the ATmega 8-bit AVR enhanced RISC
architecture. Some other Arduino models use different microcontrollers/processors.
\item \url{https://www.arduino.cc}
\item If you have your own microcontroller of a different type, such as PIC
or ARM, you are welcome to use it for your project, so long as you follow the same
requirements. Depending on department funds, we may be able to purchase a specific
microcontroller for you (make requests to the professor).
\end{itemize}
\textbf{Option 2 - Software Project:}
\begin{itemize}
\item Code must be unique to this class.
\item Must be substantial. That is, not as basic as the weekly assignments in
this class.
\item Could write a new program from scratch, in part or in whole, with
Assembly.
\item Could re-implement a program in Assembly that you previously wrote in a
high-level language.
\item Could re-implement function(s) in Assembly that are part of a larger
project you have written. This means linking Assembly code with a high-level
language. This is an opportunity to make a program more efficient. In fact, as part
of your technical report you could have results of benchmark tests based on both
versions of the program.
\end{itemize}
\noindent No matter which option you choose, the technical report must explain \
textbf{in detail} the following components from a macro (project as a whole) and
micro stance (the finer details). In a sense, you are documenting the project and
your progress throughout implementation.
\begin{itemize}
\item What?
\item Why?
\item How?
\item Challenges?
\item Solutions?
\item Explanations
\item Visualizations
\item Sources where applicable
\end{itemize}
I expect professional writing, professional quality, and professional
presentations.
\end{document}


